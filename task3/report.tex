\documentclass[a4paper,12pt]{article}
\usepackage[utf8x]{inputenc}
\usepackage[T2A]{fontenc}
\usepackage[russian,english]{babel}
\usepackage{amsmath}
\usepackage{cmap}
\usepackage{booktabs}
\usepackage{caption}
\usepackage{enumitem}
\usepackage{listings}
\usepackage{xcolor}
\usepackage{setspace}
\usepackage[left=2cm, right=1.5cm, top=2cm, bottom=2cm]{geometry}
\renewcommand{\labelenumii}{\arabic{enumi}.\arabic{enumii}.}
\lstset{
    language=C++,
    basicstyle=\small\ttfamily,
    keywordstyle=\color{blue},
    commentstyle=\color{green!40!black},
    stringstyle=\color{purple},
    numbers=left,
    numberstyle=\tiny,
    numbersep=5pt,
    breaklines=true,
    frame=single,
    backgroundcolor=\color{gray!10},
    rulecolor=\color{black!30},
    showstringspaces=false,
    extendedchars=\true, % Включение расширенных символов, включая русский текст
}
\begin{document}

\begin{center}
\hfill \break
\textbf{\large{Министерство науки и высшего образования Российской Федерации\\
Федеральное государственное автономное образовательное\\
учреждение высшего образования}}
\\
\large{\textbf{«КАЗАНСКИЙ (ПРИВОЛЖСКИЙ) ФЕДЕРАЛЬНЫЙ УНИВЕРСИТЕТ»}}\\
\hfill \break
\large{ИНСТИТУТ ВЫЧИСЛИТЕЛЬНОЙ МАТЕМАТИКИ\\ И ИНФОРМАЦИОННЫХ ТЕХНОЛОГИЙ}\\
\hfill \break
\large{Кафедра прикладной математики и искусственого интеллекта}\\
\hfill\break
\hfill \break
\large{Направление подготовки: 01.03.04 – Прикладная математика}\\
\hfill \break
\hfill \break
\textbf{\large{ОТЧЁТ}}\\
\large{По дисциплине <<Численные методы>>}\\
\large{на тему:}\\
\large{<<Система линейных алгебраических уравнений>>}\\
\hfill \break
\hfill \break
\end{center}

\hfill \break
\begin{flushright}
			
    \large{Выполнил:}
    
    \large{студент группы 09-222}
    
    \large{Романов И. И.}
    
    \large{Проверил:}
    
    \large{ассистент Глазырина О.В.}
    
\end{flushright}
\vfill
\begin{center} \large{Казань, 2024 год} \end{center}
\thispagestyle{empty}
 

\newpage
\begin{center}
\renewcommand{\contentsname}{Содержание}
\fontsize{14}{1.15}\selectfont
\mdseries\selectfont{\tableofcontents}
\newpage
\end{center}
\setlength{\parindent}{1.25cm}
\newpage
\selectfont\onehalfspacing{
\section{Постановка задачи}
\hspace{1.25cm}Решить систему линейных алгебраических уравнений:
\begin{equation}\label{eq:main_sys}
    \begin{cases}
        &(a_1 + a_2 + h^2g_1)y_1 - a_2y_2 = f_1h^2,\\
        &\hspace{1cm}\dots \quad \dots \quad \dots \quad \dots\\
        &-a_iy_{i-1} + (a_i + a_{i+1} + h^2g_i)y_i - a_{i+1}y_{i+1} = f_ih^2,\\
        &\hspace{1cm}\dots \quad \dots \quad \dots \quad \dots\\
        &(a_{n-1} + a_{n} + h^2g_{n-1})y_{n-1} - a_{n-1}y_{n-2} = f_{n-1}h^2.\\
    \end{cases}
\end{equation}

Здесь $a_i = p(ih),\;g_i = q(ih),\;f_i = f(ih),\;f(x) = -(p(x)u'(x))' + q(x)u(x),\;\\h = 1/n,\;p,\;q,\;u\;~-$ заданные функции.

Данную систему решить методом прогоки и итерационными методами:
\begin{enumerate}[label = \arabic*.]
    \item метод Зейделя.
    \item метод нижней релаксации.
    \item метод наискорейшего спуска.
\end{enumerate}

Во всех итерационных методах вычисления продолжать до выполнения условия:
\begin{equation*}
    \max_{1 \le i \le n - 1} \left|r_{i}^{k}\right| \le \varepsilon,
\end{equation*}
$r\;~-$ вектор невязки, $\varepsilon\;~-$ заданное число.

\textbf{Исходные данные:} $n = 10,\; n = 50,\;\varepsilon = h^3,\;u(x) = x^{\alpha}(1-x)^{\beta},\;\\p(x) = 1 + x^{\gamma},\;g(x) = x + 1,\;\alpha = 1,\;\beta = 3,\;\gamma = 2.$

Для сравнения результатов вычисления составим соответсвтенные таблицы и подведём выводы.
\section{Ход работы}
\subsection{Метод прогонки}
\hspace*{1.25cm} Метод прогонки состоит из двух этапов: прямой ход (определение прогоночных коэффициентов), 
обратных ход (вычисление неизвестных $y_i$).

Основным преимуществом является экономичность $~-$ максимально использование структуры исходной системы.

К недостаткам же можно отнести то, что с каждой итерацией накапливается ошибка округления.

% Запишем систему \eqref{eq:main_sys} в следующем виде:
% \begin{equation}
%     \begin{cases}
%         &-b_1y_1 + c_1y_2 = f_1,\\
%         &\hspace{1cm}\dots \quad \dots \quad \dots \quad \dots\\
%         &a_iy_{i-1} -b_iy_i + c_iy_{i+1} = f_i,\\
%         &\hspace{1cm}\dots \quad \dots \quad \dots \quad \dots\\
%         &b_ny_n - a_{n-1}y_{n-1} = f_{n-1}.\\
%     \end{cases}
% \end{equation}

Реализуем прямой ход метода и найдём прогоночные коэффициенты:
\begin{equation}
    \begin{cases}
        \alpha_1 = \dfrac{a_1}{a_0 + a_1 + h^2g_1},\\
        \alpha_{i+1} = \dfrac{a_{i+1}}{a_i + a_{i+1} + h^2g_i - \alpha_ia_i}, \quad i = \overline{2, n - 1};\\
    \end{cases}
\end{equation}
\begin{equation}
    \begin{cases}
        \beta_1 = \dfrac{f_0h^2}{a_0 + a_1 + h^2g_1},\\
        \beta_{i+1} = \dfrac{f_ih^2 + \beta_ia_i}{a_i + a_{i+1} + h^2g_i - \alpha_ia_i}, \quad i = \overline{2, n - 1};\\
    \end{cases}
\end{equation}

Обратных ход метода:
\begin{equation}
    \begin{cases}
        y_n = \beta_{n+1};\\
        y_i = \alpha_{i+1}y_{i+1} + \beta_{i+1}, \quad i = \overline{n-1, 0};\\
    \end{cases}
\end{equation}

Формулы (2-4) являются методом Гаусса, записанным применительно трёх\-диаго\-наль\-ной системы уравнений. 
Метод может быть реализован только в случае, когда в формулах (3) и (4) все знаменатели отличны от нуля, 
то есть условие выполняется, когда матрица системы (2) имеет диагональное препобладание.
Это означает выполняется условие:\\$|c_0| < |b_0|,\;|a_n|<|b_n|,\;|a_i| + |c_i| < |b_i|,\;i = 1,\dots n-1.$

Проделаем вычисления и составим таблицу для $n = 10$ (Таблица 1), для $n = 50$ (Таблица 2):
\begin{table}[h]
    \centering
    \begin{tabular}{|c|c|c|c|}
        \hline
        $ih$ & $y_i$ & $u(ih)$ & $\left|y_i - u(ih)\right|$\\
        \hline
        0,0 & 0,000000  & 0,000000  & 0,000000  \\           
        \hline
        0,1 & 0,071338  & 0,072900  & 0,001562  \\
        \hline
        0,2 & 0,099583  & 0,102400  & 0,002817  \\
        \hline
        0,3 & 0,099491  & 0,102900  & 0,003409  \\
        \hline
        0,4 & 0,083074  & 0,086400  & 0,003326  \\
        \hline
        0,5 & 0,059738  & 0,062500  & 0,002762  \\
        \hline
        0,6 & 0,036421  & 0,038400  & 0,001979  \\
        \hline
        0,7 & 0,017694  & 0,018900  & 0,001206  \\
        \hline
        0,8 & 0,005825  & 0,006400  & 0,000575  \\
        \hline
        0,9 & 0,000812  & 0,000900  & 0,000088  \\
        \hline
        1,0 & 0,000384  & 0,000000  & 0,000384  \\
        \hline
    \end{tabular}
    \caption*{\small{Таблица 1 - значения метода прогонки для $n = 10$}}
\end{table}
\clearpage
\begin{table}[h]
    \centering
    \begin{tabular}{|c|c|c|c|}
        \hline
        $ih$ & $y_i$ & $u(ih)$ & $\left|y_i - u(ih)\right|$\\
        \hline
        0,0 & 0,000000  & 0,000000  & 0,000000  \\           
        \hline
        0,10 & 0,072724  & 0,072900  & 0,000176  \\
        \hline
        0,20 & 0,083074  & 0,086400  & 0,003326  \\
        \hline
        0,30 & 0,102460  & 0,102900  & 0,000440  \\
        \hline
        0,40 & 0,085983  & 0,086400  & 0,000417  \\
        \hline
        0,50 & 0,062186  & 0,062500  & 0,000314  \\
        \hline
        0,60 & 0,038220  & 0,038400  & 0,000180  \\
        \hline
        0,70 & 0,018838  & 0,018900  & 0,000062  \\
        \hline
        0,80 & 0,006408  & 0,006400  & 0,000008  \\
        \hline
        0,90 & 0,000925  & 0,000900  & 0,000025  \\
        \hline
        1,0 & 0,000011  & 0,000000  & 0,000011  \\
        \hline
    \end{tabular}
    \caption*{\small{Таблица 2 - значения метода прогонки для $n = 50$}}
\end{table}
\subsection{Метод Зейделя}
\hspace*{1.25cm}Для больших систем вида $Ax=b$ предпочтительнее оказываются итерационные методы. 
Основная идея данных методов состоит в построении последовательности векторов $x^k, \; k=1,2,\dots,$ 
сходящейся к решению исходной системы. За приближенное решение принимается вектор $x^k$ при достаточно большом $k$.

Будем считать, что все диагональные элементы матрицы $А$ из полной системы $Ax=b$ отличны от нуля. 
Представим эту систему, разрешая каждое уранвение относительно переменной, стоящей на главной диагонали: 
\begin{equation}
x_i = - \sum\limits_{j=1}^{i-1} \frac{a_{ij}}{a_{ii}}x_j - \sum\limits_{j=i+1}^n \frac{a_{ij}}{a_{ii}}x_j+\frac{b_i}{a_{ii}}, \quad i = \overline{1,n}.
 \label{5}
\end{equation}

Выберем некоторое начальное приближение $x^0=(x_1^0, x_2^0, {\dots}, x_n^0)^T$. 
Построим последова\-тель\-ность векторов $x^1, x^2, {\dots},$ определяя вектор $x^{k+1}$ по уже найденному вектору $x^k$ при помощи соотношения: 
\begin{equation}
x_i^{k+1}= - \sum\limits_{j=1}^{i-1} \frac{a_{ij}}{a_{ii}}x_j^k - \sum\limits_{j=i+1}^n \frac{a_{ij}}{a_{ii}}x_j^k+\frac{b_i}{a_{ii}}, \quad i = \overline{1,n}. 
 \label{6}
\end{equation}

Формула (\ref{6}) определяют итерационный метод решения системы (\ref{5}), называемый мето\-дом Якоби или методом простой итерации.

Метод Якоби допускает естественную модификацию: при вычислении $x_i^{k+1}$ 
будем использовать уже найденные компоненты вектора $x^{k+1}$, то есть $x_1^{k+1},\;x_2^{k+1},\;\dots,\;x_{i-1}^{k+1}$. 
В резултате приходим к итерационному методу Зейделя:
\begin{equation}
    x_i^{k+1}= - \sum\limits_{j=1}^{i-1} \frac{a_{ij}}{a_{ii}}x_j^{k+1} - \sum\limits_{j=i+1}^n \frac{a_{ij}}{a_{ii}}x_j^k+\frac{b_i}{a_{ii}}, \quad i = \overline{1,n}. 
     \label{7}
\end{equation}

Запишем формулу (\ref{7}) для нашей системы \eqref{eq:main_sys}:
\begin{equation}
    y_i^{k+1}=  -\sum\limits_{j=1}^{i-1}\frac{a_j}{a_i+a_{i+1}+h^2 g_i}y_{i-1}^k \; - \;\sum\limits_{j=i+1}^{n}\frac{a_{j}}{a_i+a_{i+1}+h^2 g_i}y_{i+1}^k\; +\; \frac{f_i h^2}{a_i+a_{i+1}+h^2 g_i},$$$$
    i=\overline{1, n-1};
     \label{8}
\end{equation}

Вычисления продолжаем, пока не выполнится условие:
$$\max\limits |r_i^k| \leq \varepsilon,$$
где $r^k ~-$ вектор невязки для $k$-той итерации $r^k=Ay^k-f, \quad \varepsilon=h^3.$

Составим таблицы вычесленных результатов для $n = 10$, для $n = 50$, 
в которых будем сравнивать значения метода прогонки и метода Зейделя для точки $i,\; i = \overline{0, n - 1}$, 
найдём модуль их разности и значение $k$, при котором была достигнута необходимая точность.
\begin{table}[h]
    \centering
    \begin{tabular}{|c|c|c|c|c|}
        \hline
        $i$ & $y_i$ & $y_i^k$ & $\left|y_i - y_i^k\right|$ & $k$\\
        \hline
        0 &  0,071338 &  0,068591 &  0,002747 & 19 \\ \hline
        1 &  0,099583 &  0,094675 &  0,004908 & 19 \\ \hline
        2 &  0,099491 &  0,093153 &  0,006339 & 19 \\ \hline
        3 &  0,083074 &  0,076065 &  0,007009 & 19 \\ \hline
        4 &  0,059738 &  0,052761 &  0,006977 & 19 \\ \hline
        5 &  0,036421 &  0,030057 &  0,006363 & 19 \\ \hline
        6 &  0,017694 &  0,012371 &  0,005322 & 19 \\ \hline
        7 &  0,005825 &  0,001806 &  0,004019 & 19 \\ \hline
        8 &  0,000812 & -0,001799 &  0,002611 & 19 \\ \hline
        9 &  0,000384 & -0,000850 &  0,001234 & 19 \\ \hline
    \end{tabular}
    \caption*{\small{Таблица 3 - значения метода Зейделя для $n = 10$}}
\end{table}
\begin{table}[h]
    \centering
    \begin{tabular}{|c|c|c|c|c|}
        \hline
        $i$ & $y_i$ & $y_i^k$ & $\left|y_i - y_i^k\right|$ & $k$\\
        \hline
        0 &  0,018796 &  0,018695 &  0,000101 & 851 \\ \hline
        4 &  0,072724 &  0,072233 &  0,000491 & 851 \\ \hline
        9 &  0,102046 &  0,101135 &  0,000912 & 851 \\ \hline
       14 &  0,102460 &  0,101242 &  0,001218 & 851 \\ \hline
       19 &  0,085983 &  0,084599 &  0,001384 & 851 \\ \hline
       24 &  0,062186 &  0,060785 &  0,001402 & 851 \\ \hline
       29 &  0,038220 &  0,036937 &  0,001283 & 851 \\ \hline
       34 &  0,018838 &  0,017783 &  0,001055 & 851 \\ \hline
       39 &  0,006408 &  0,005658 &  0,000750 & 851 \\ \hline
       44 &  0,000925 &  0,000518 &  0,000408 & 851 \\ \hline
       49 &  0,000011 & -0,000054 &  0,000065 & 851 \\ \hline
    \end{tabular}
    \caption*{\small{Таблица 4 - значения метода Зейделя для $n = 50$}}
\end{table}
\clearpage
\subsection{Метод верхней релаксации}
\hspace{1.25cm}Во многих ситуациях существенного ускорения сходимости можно добиться за счет введения так называемого итерационного параметра. 
Рассмотрим итерационный процесс:
\begin{equation}
    x_i^{k+1}=(1-\omega)x_i^k+\omega \bigg( - \sum\limits_{j=1}^{i-1} \frac{a_{ij}}{a_{ii}}x_j^{k+1}- \sum\limits_{j=i+1}^{n} \frac{a_{ij}}{a_{ii}}x_j^{k}+\frac{b_i}{a_{ii}}\bigg),
    $$$$ i=1,2, \dots, n, \quad k=0,1, \dots \; .
    \label{9}
\end{equation}

Этот метод называется методом релаксации -- одним из наиболее эффективных и широко используемых итерационных 
методов для решения систем линейных алгебраичес\-ких уравнений. Значение $\omega $ -- называется релаксационным параметром. 
При $\omega = 1$ метод переходит в метод Зейделя. При $\omega \in (1,2)$ -- это метод верхней релаксации, при $\omega \in (0,1)$ -- метод нижней релаксации. 
Ясно, что по затратам памяти и объему вычислений на каждом шаге итераций метод релаксации не отличается от метода Зейделя.

Преобразуем формулу \eqref{9} относительно нашей системы:
\begin{equation}
    y_i^{k+1}=(1-\omega)y_i^k+\omega  \bigg(-\sum\limits_{j=1}^{i-1}\frac{a_j}{a_i+a_{i+1}+h^2 g_i}y_{i-1}^k \; - \;\sum\limits_{j=i+1}^{n}\frac{a_{j}}{a_i+a_{i+1}+h^2 g_i}y_{i+1}^k\; +\; \frac{f_i h^2}{a_i+a_{i+1}+h^2 g_i}\bigg),$$$$
    i=\overline{1, n-1};
    \label{10}
\end{equation}

Параметр $\omega$ следует выбирать так, чтобы метод релаксации сходился наиболее быстро. 
Нужно отметить, что оптимальный параметр для метода верхней релаксации лежит вблизи 1,8. 
Заполним таблицы, в которых приведём значения параметра $\omega$ и количство итераций $k$:
\begin{table}[h]
    \centering
    \begin{tabular}{|c|c|}
        \hline
        $\omega$ & $k$\\
        \hline
        1.1 & 15\\ \hline
        1.2 & 12\\ \hline
        1.3 & 9\\ \hline
        1.4 & 7\\ \hline
        1.5 & 6\\ \hline
        1.6 & 8\\ \hline
        1.7 & 11\\ \hline
        1.8 & 18\\ \hline
        1.9 & 34\\ \hline
    \end{tabular}
    \caption*{\small{Таблица 5 - значения $\omega$ и соответствующие значения $k$ для $n = 10$}}
\end{table}
\clearpage
\begin{table}[h]
    \centering
    \begin{tabular}{|c|c|}
        \hline
        $\omega$ & $k$\\
        \hline
        1.02 &  818\\ \hline
        1.14 &  639\\ \hline
        1.30 &  452\\ \hline
        1.38 &  375\\ \hline
        1.54 &  245\\ \hline
        1.62 &  189\\ \hline
        1.78 &  93\\ \hline
        1.86 &  51\\ \hline
        1.94 &  102\\ \hline
    \end{tabular}
    \caption*{\small{Таблица 6 - значения $\omega$ и соответствующие значения $k$ для $n = 50$}}
\end{table}

Для вычислений выберем $\omega = 1,86$. Составим таблицы результатов для $n = 10$, $n = 50$,
в которых будем сравнивать значения метода прогонки и метода верхней релаксации для точки $i,\; i = \overline{0, n-1}$,
найдём модуль их разности и значение $k$:
\begin{table}[h]
    \centering
    \begin{tabular}{|c|c|c|c|c|}
        \hline
        $i$ & $y_i$ & $y_i^k$ & $\left|y_i - y_i^k\right|$ & $k$\\
        \hline
        0 &  0,071338 &  0,071702 &  0,000364 & 18 \\ \hline
        1 &  0,099583 &  0,099998 &  0,000416 & 18 \\ \hline
        2 &  0,099491 &  0,099766 &  0,000275 & 18 \\ \hline
        3 &  0,083074 &  0,083252 &  0,000178 & 18 \\ \hline
        4 &  0,059738 &  0,059116 &  0,000622 & 18 \\ \hline
        5 &  0,036421 &  0,035560 &  0,000861 & 18 \\ \hline
        6 &  0,017694 &  0,016902 &  0,000792 & 18 \\ \hline
        7 &  0,005825 &  0,005242 &  0,000583 & 18 \\ \hline
        9 &  0,000384 &  0,000246 &  0,000138 & 18 \\ \hline
    \end{tabular}
    \caption*{\small{Таблица 7 - значения метода верхней релаксации для $n = 10$}}
\end{table}
\begin{table}[h]
    \centering
    \begin{tabular}{|c|c|c|c|c|}
        \hline
        $i$ & $y_i$ & $y_i^k$ & $\left|y_i - y_i^k\right|$ & $k$\\
        \hline
        0 &  0,018796 &  0,018671 &  0,000125 & 51 \\ \hline
        9 &  0,102046 &  0,101108 &  0,000938 & 51 \\ \hline
       14 &  0,102460 &  0,101328 &  0,001132 & 51 \\ \hline
       24 &  0,062186 &  0,061123 &  0,001063 & 51 \\ \hline
       29 &  0,038220 &  0,037341 &  0,000880 & 51 \\ \hline
       34 &  0,018838 &  0,018184 &  0,000653 & 51 \\ \hline
       39 &  0,006408 &  0,005988 &  0,000420 & 51 \\ \hline
       44 &  0,000925 &  0,000719 &  0,000206 & 51 \\ \hline
    \end{tabular}
    \caption*{\small{Таблица 8 - значения метода верхней релаксации для $n = 50$}}
\end{table}
\clearpage
\subsection{Метод наискорейшего спуска}
\hspace{1.25cm}Существуют итерационные методы, позволяющие за счет некоторой дополнительной работы на 
каждом шаге итераций автоматически настраиваться на оптимальную скорость сходимости. К их числу относятся методы, основанные 
на замене системы \eqref{6} эквивалентной задачей минимизации некоторого функционала.

Опишем итерационный метод наискорейшего спуска. Будем двигаться из точки началь\-ного приближения $x^0$ в направлении наибыстрейшего убывания функционала $F$, 
то есть следующее приближение будем разыскивать так: $x^1=x^0 - \tau \mathrm { grad } F(x^0).$ Формула:
\begin{equation}
    F'_{x_i}(x) = 2\sum\limits_{j=1}^{n}a_{ij}x_j - 2b_i;
    \label{11}
\end{equation}
, которая является производной функции $F(x)$ по переменной $x_i$, показывает, что $\mathrm { grad } F(x^0)=2(Ax^0-b).$ 
Вектор $r_0 = Ax^0-b$ принято называть невязкой. Для сокращения записей удобно обозначить $2\tau$ вновь через $\tau.$ 
Таким образом, $x^1=x^0-\tau r^0.$

Параметр $\tau$ выберем так, чтобы значение $F(x^1)$ было минимальным. Получим $F(x^1)=F(x^0-\tau r^0)=F(x^0)-2\tau(r^0,r^0)+\tau^2(Ar^0,r^0),$ 
следовательно, минимум $F(x^1)$ достигается при $\tau =\tau_*=\displaystyle\frac{(r^0, r^0)}{(Ar^0, r^0)}.$

Таким образом, мы пришли к следующему итерационнму методу:
\begin{equation}
x^{k+1}=x^k-\tau_* r^k, \quad r^k=Ax^k-b, \quad \tau_*=\displaystyle\frac{(r^k, r^k)}{(Ar^k, r^k)}, \; k=0,1, \dots \;.
 \label{12}
\end{equation}

Метод ($12$) называют методом наискорейшего спуска. По сравнению с методом простой итерации этот метод требует на каждом шаге итераций 
проведения дополнительной работы по вычислению параметра $\tau_*.$ Вследствие этого происходит адаптация к оптимальной скорости сходимости.

Составим таблицы результатов для $n = 10$, $n = 50$,
в которых будем сравнивать значения метода прогонки и метода верхней релаксации для точки $i,\; i = \overline{0, n-1}$,
найдём модуль их разности и значение $k$:
\begin{table}[h]
    \centering
    \begin{tabular}{|c|c|c|c|c|}
        \hline
        $i$ & $y_i$ & $y_i^k$ & $\left|y_i - y_i^k\right|$ & $k$\\
        \hline
        0 &  0,071338 &  0,068795 &  0,002543 & 22 \\ \hline
        1 &  0,099583 &  0,095015 &  0,004568 & 22 \\ \hline
        2 &  0,099491 &  0,093633 &  0,005858 & 22 \\ \hline
        3 &  0,083074 &  0,076716 &  0,006359 & 22 \\ \hline
        4 &  0,059738 &  0,053639 &  0,006099 & 22 \\ \hline
        5 &  0,036421 &  0,031081 &  0,005340 & 22 \\ \hline
        6 &  0,017694 &  0,013500 &  0,004194 & 22 \\ \hline
        7 &  0,005825 &  0,002806 &  0,003019 & 22 \\ \hline
        8 &  0,000812 & -0,000835 &  0,001647 & 22 \\ \hline
        9 &  0,000384 & -0,000349 &  0,000733 & 22 \\ \hline
    \end{tabular}
    \caption*{\small{Таблица 9 - значения метода наискорейшего спуска для $n = 10$}}
\end{table}
\begin{table}[h]
    \centering
    \begin{tabular}{|c|c|c|c|c|}
        \hline
        $i$ & $y_i$ & $y_i^k$ & $\left|y_i - y_i^k\right|$ & $k$\\
        \hline
        0 &  0,018796 &  0,018713 &  0,000083 & 617 \\ \hline
        5 &  0,081561 &  0,081089 &  0,000472 & 617 \\ \hline
       10 &  0,104022 &  0,103249 &  0,000772 & 617 \\ \hline
       15 &  0,100175 &  0,099229 &  0,000946 & 617 \\ \hline
       20 &  0,081546 &  0,080564 &  0,000982 & 617 \\ \hline
       25 &  0,057220 &  0,056321 &  0,000900 & 617 \\ \hline
       30 &  0,033867 &  0,033129 &  0,000738 & 617 \\ \hline
       35 &  0,015762 &  0,015226 &  0,000535 & 617 \\ \hline
       40 &  0,004798 &  0,004465 &  0,000332 & 617 \\ \hline
       45 &  0,000495 &  0,000343 &  0,000152 & 617 \\ \hline
       46 &  0,000224 &  0,000108 &  0,000116 & 617 \\ \hline
       47 &  0,000080 & -0,000005 &  0,000085 & 617 \\ \hline
       48 &  0,000022 & -0,000033 &  0,000056 & 617 \\ \hline
       49 &  0,000011 & -0,000015 &  0,000026 & 617 \\ \hline
    \end{tabular}
    \caption*{\small{Таблица 10 - значения метода наискорейшего спуска для $n = 50$}}
\end{table}
\clearpage
\section{Выводы}
\hspace{1.25cm}В процессе выполнения работы были изучены методы решения заданной системы линейных алгебраических уравнений вида:
\begin{equation*}
    \begin{cases}
        &(a_1 + a_2 + h^2g_1)y_1 - a_2y_2 = f_1h^2,\\
        &\hspace{1cm}\dots \quad \dots \quad \dots \quad \dots\\
        &-a_iy_{i-1} + (a_i + a_{i+1} + h^2g_i)y_i - a_{i+1}y_{i+1} = f_ih^2,\\
        &\hspace{1cm}\dots \quad \dots \quad \dots \quad \dots\\
        &(a_{n-1} + a_{n} + h^2g_{n-1})y_{n-1} - a_{n-1}y_{n-2} = f_{n-1}h^2.\\
    \end{cases}
\end{equation*}
при помощи:
\begin{enumerate}[label = \arabic*.]
    \item метод прогонки.
    \item метод Зейделя.
    \item метод нижней релаксации.
    \item метод наискорейшего спуска.
\end{enumerate}

После вычисления результатов решения системы линейных алгебраических уравнений можно сделать вывод, 
что наилучшим методом дря решения является метод верхней релакса\-ции при итерационном параметре $\omega = 1,86$. 
Данный метод показывается наилучшие результа\-ты вычисления корней системы за наименьшее количество итераций.
}
\clearpage
\section{Cписок литературы}
\begin{enumerate}
    \item Глазырина Л.Л., Карчевский М.М. Численные методы: учебное пособие. — Казань: Казан.
    ун-т, 2012. — 122 
    \item Глазырина Л.Л.. Практикум по курсу «Численные методы». Решение
    систем линейных уравнений: учеб. пособие. — Казань: Изд-во Казан. ун-та, 2017. — 52 с.
\end{enumerate}
\clearpage
\section{Листинг программы}
\lstinputlisting[language=C++]{./header.hpp}
\lstinputlisting[language=C++]{./main.cpp}
\end{document}