\documentclass[14pt]{article}
\usepackage[utf8x]{inputenc}
\usepackage[T2A]{fontenc}
\usepackage[russian,english]{babel}
\usepackage{amsmath}
\usepackage{cmap}
\usepackage{booktabs}
\usepackage{caption}
\usepackage[a4paper, left=2cm, right=2cm, top=2cm, bottom=2cm]{geometry}

\begin{document}
 
\begin{center}
\hfill \break
\textbf{\normalsize{Министерство науки и высшего образования Российской Федерации\\
Федеральное государственное автономное образовательное\\
учреждение высшего образования}}
\\
\normalsize{\textbf{«КАЗАНСКИЙ (ПРИВОЛЖСКИЙ) ФЕДЕРАЛЬНЫЙ УНИВЕРСИТЕТ»}}\\
\hfill \break
\normalsize{ИНСТИТУТ ВЫЧИСЛИТЕЛЬНОЙ МАТЕМАТИКИ\\ И ИНФОРМАЦИОННЫХ ТЕХНОЛОГИЙ}\\
 \hfill \break
\normalsize{Кафедра прикладной математики}\\
\hfill\break
\hfill \break
\normalsize{Направление подготовки: 01.03.04 – Прикладная математика}\\
\hfill \break
\hfill \break
\textbf{ОТЧЁТ}\\
\normalsize{По дисциплине <<Численные методы>>}\\
\normalsize{на тему:}\\
\normalsize{<<Вычисление интеграла с помощью квадратурных формул>>}\\
\hfill \break
\hfill \break
\end{center}

\hfill \break
\textbf{Выполнил:}\\Романов И.И. 09-222 группа.
\hfill \break
\hfill \break
\textbf{Руководитель:}\\Глазырина О.В.
\\
\\
\\
\\
\\
\\
\\
\\
\\
\\
\\
\\
\\
\\
\\
\\
\\
\\
\\
\\
\\
\\
\\
\\
\\
\\
\\
\\
\\
\\
\begin{center} Казань 2024 \end{center}
\thispagestyle{empty}
 

\newpage
\begin{center}
\renewcommand{\contentsname}{Содержание}
\tableofcontents
\newpage
\end{center}
\newpage
\section{Постановка задачи}
\hspace{5mm}Необходимо изучить и сравнить различные способы приближённого вычисления функции ошибок
\begin{center}
    $\text{erf}(x) =\displaystyle\frac{2}{\sqrt{\pi}} \displaystyle\int\limits_{0}^{x} e^{-t^2} dt$
\end{center}
\begin{enumerate}
    \item Протабулировать erf(x) на отрезке [a,b] с шагом h и точностью $ \varepsilon$, основываясь на ряде Тейлора,\\
     предварительно вычислив его
     \begin{center}
        $\text{erf}(x) =\displaystyle\frac{2}{\sqrt{\pi}} \displaystyle\sum\limits_{n=1}^{\infty}(-1)^n\displaystyle\frac{x^{2n+1}}{n!(2n+1)} $
    \end{center}
    где a = 0, b = 2, h = 0.2, $\varepsilon = 10^{-6}$. Получив таким образом таблицу из 11 точек
    \begin{tabbing}
        $x_0$ \= $x_1$ \= $x_2$ \= \dots\\
        $f_0$ \> $f_1$ \> $f_2$ \> \dots\\
    \end{tabbing}
    $f_i = \text{erf}(x_i)$, $x_i = a + i \cdot h$, $i = 0,\dots,n.$
    \item {Вычислить erf(x) при помощи пяти составных квадратурных формул:
        \begin{enumerate}
            \item {
                Формула Правых прямоугольников:
                \begin{center}
                    $J_N(x) = \displaystyle\sum_{i=2}^{n+1}h \cdot g(x_i)$, $h=(x_{i+1} - x_i)$
                \end{center}
                }
            \item {
                Формула Центральных прямоугольников:
                \begin{center}
                    $J_N(x) = \displaystyle\sum_{i=1}^{n}h \cdot g\left(\displaystyle\frac{x_i + x_{i+1}}{2}\right)$  
                \end{center}
                }
            \item {
                Формула трапеции:
                \begin{center}
                    $J_N(x) = \displaystyle\sum_{i=1}^{n}h \cdot \displaystyle\frac{g(x_i)+g(x_{i+1})}{2}$
                \end{center}
                }
            \item {
                Формула Симпсона:
                \begin{center}
                    $J_N(x) = \displaystyle\sum_{i=1}^{n}\displaystyle\frac{h}{6} \cdot \left[g(x_i)+4g\left(
                        \displaystyle\frac{x_i + x_{i+1}}{2}
                    \right) + g(x_{i+1}) \right]$
                \end{center}
                }
            \item {
                Формула Гаусса:
                \begin{center}
                    $J_N(x) = \displaystyle\sum_{i=1}^{n}\displaystyle\frac{h}{2} \cdot \left[g\left(
                        x_i + \displaystyle\frac{h}{2}\left(1 - \displaystyle\frac{1}{\sqrt{3}} \right)
                    \right) +
                    g\left(
                        x_i + \displaystyle\frac{h}{2}\left(1 + \displaystyle\frac{1}{\sqrt{3}} \right)
                    \right)
                    \right]$
                \end{center}
                }
        \end{enumerate}
    }
\end{enumerate}

Вычисления проводятся от начала интегрирования до каждой из 11 точек, увеличивая количество разбиений между точками в 2 раза до тех пор, пока погрешность больше $\varepsilon$.
\newpage
\section{Ход работы}
Выберем точки на отрезке [a,b] с шагом h.
\begin{center}
    $x_i = a + i \cdot h$.
\end{center}
Для каждой точки $x_i$ найдём значение $f(x_i)$ и составим таблицу результатов (Таблица 1).
\begin{table}[h]
    \centering
    \begin{tabular}{|c|c|}
        \hline
        $x_i$ & $f(x_i)$\\
        \hline
        0,0 & 0,0\\
        \hline
    \end{tabular}
    \caption*{Таблица 1 - точки $x_i$ и значения ряда Тейлора $f(x_i)$}
\end{table}
\newpage
\section{Выводы}

\newpage
\section{Листинг программы}


\end{document}