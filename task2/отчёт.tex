\documentclass[14pt]{article}
\usepackage[utf8x]{inputenc}
\usepackage[T2A]{fontenc}
\usepackage[russian,english]{babel}
\usepackage{amsmath}
\usepackage{cmap}
\usepackage{booktabs}
\usepackage{caption}
\usepackage{enumitem}
\usepackage{listings}
\usepackage{xcolor}
\usepackage[a4paper, left=2cm, right=2cm, top=2cm, bottom=2cm]{geometry}
\renewcommand{\labelenumii}{\arabic{enumi}.\arabic{enumii}.}
\lstset{
    language=C++,
    basicstyle=\small\ttfamily,
    keywordstyle=\color{blue},
    commentstyle=\color{green!40!black},
    stringstyle=\color{purple},
    numbers=left,
    numberstyle=\tiny,
    numbersep=5pt,
    breaklines=true,
    frame=single,
    backgroundcolor=\color{gray!10},
    rulecolor=\color{black!30},
    showstringspaces=false,
    extendedchars=\true, % Включение расширенных символов, включая русский текст
}
\begin{document}

\begin{center}
\hfill \break
\textbf{\large{Министерство науки и высшего образования Российской Федерации\\
Федеральное государственное автономное образовательное\\
учреждение высшего образования}}
\\
\large{\textbf{«КАЗАНСКИЙ (ПРИВОЛЖСКИЙ) ФЕДЕРАЛЬНЫЙ УНИВЕРСИТЕТ»}}\\
\hfill \break
\large{ИНСТИТУТ ВЫЧИСЛИТЕЛЬНОЙ МАТЕМАТИКИ\\ И ИНФОРМАЦИОННЫХ ТЕХНОЛОГИЙ}\\
 \hfill \break
\large{Кафедра прикладной математики и искусственого интеллекта}\\
\hfill\break
\hfill \break
\large{Направление подготовки: 01.03.04 – Прикладная математика}\\
\hfill \break
\hfill \break
\textbf{\large{ОТЧЁТ}}\\
\large{По дисциплине <<Численные методы>>}\\
\large{на тему:}\\
\large{<<Вычисление интеграла с помощью квадратурных формул>>}\\
\hfill \break
\hfill \break
\end{center}

\hfill \break
\large{\textbf{Выполнил:}\\Романов И.И. 09-222 группа.}
\hfill \break
\hfill \break
\large{\textbf{Руководитель:}\\Глазырина О.В.}
\\
\\
\\
\\
\\
\\
\\
\\
\\
\\
\\
\\
\\
\\
\\
\\
\\
\\
\\
\\
\\
\\
\\
\begin{center} Казань 2024 \end{center}
\thispagestyle{empty}
 

\newpage
\begin{center}
\renewcommand{\contentsname}{Содержание}
\tableofcontents
\newpage
\end{center}
\newpage
\section{Постановка задачи}
\hspace{5mm}Необходимо изучить и сравнить различные способы приближённого вычисления функции ошибок
\begin{equation}
    \text{erf}(x) =\displaystyle\frac{2}{\sqrt{\pi}} \displaystyle\int\limits_{0}^{x} e^{-t^2} dt
\end{equation}
\begin{enumerate}
    \item Протабулировать erf(x) на отрезке [a,b] с шагом h и точностью $ \varepsilon$, основываясь на ряде Тейлора,
     предварительно вычислив его
     \begin{equation}
        \text{erf}(x) =\displaystyle\frac{2}{\sqrt{\pi}} \displaystyle\sum\limits_{n=1}^{\infty}(-1)^n\displaystyle\frac{x^{2n+1}}{n!(2n+1)}
     \end{equation}
    где $a = 0, b = 2, h = 0.2$, $\varepsilon = 10^{-6}$. Получив таким образом таблицу из 11 точек
    \begin{tabbing}
        $x_0$ \= $x_1$ \= $x_2$ \= \dots\\
        $f_0$ \> $f_1$ \> $f_2$ \> \dots\\
    \end{tabbing}
    $f_i = \text{erf}(x_i), \quad x_i = a + i   h, \quad i = 0,\dots,n.$
    \item {Вычислить erf(x) при помощи пяти составных квадратурных формул при $h=(x_{i+1} - x_i)$:
        \begin{enumerate}
            \item {
                Формула правых прямоугольников:
                \begin{equation}
                    J_N(x) = \displaystyle\sum_{i=1}^{n}h g(x_i)
                \end{equation}
                }
            \item {
                Формула центральных прямоугольников:
                \begin{equation}
                    J_N(x) = \displaystyle\sum_{i=1}^{n}h g\left(\displaystyle\frac{x_i + x_{i+1}}{2}\right) 
                \end{equation}
                }
            \item {
                Формула трапеции:
                \begin{equation}
                    J_N(x) = \displaystyle\sum_{i=1}^{n}h \displaystyle\frac{g(x_i)+g(x_{i+1})}{2}
                \end{equation}
                }
            \item {
                Формула Симпсона:
                \begin{equation}
                    J_N(x) = \displaystyle\sum_{i=1}^{n}\displaystyle\frac{h}{6} \left[g(x_i)+4g\left(
                        \displaystyle\frac{x_i + x_{i+1}}{2}
                    \right) + g(x_{i+1}) \right]
                \end{equation}
                }
            \item {
                Формула Гаусса:
                \begin{equation}
                    J_N(x) = \displaystyle\sum_{i=1}^{n}\displaystyle\frac{h}{2} \left[g\left(
                        x_i + \displaystyle\frac{h}{2}\left(1 - \displaystyle\frac{1}{\sqrt{3}} \right)
                    \right) +
                    g\left(
                        x_i + \displaystyle\frac{h}{2}\left(1 + \displaystyle\frac{1}{\sqrt{3}} \right)
                    \right)
                    \right]
                \end{equation}
                }
        \end{enumerate}
    }
\end{enumerate}

Вычисления проводятся от начала интегрирования до каждой из 11 точек, увеличивая количество разбиений между точками в 2 раза до тех пор, пока погрешность больше $\varepsilon$.
\newpage
\section{Ход работы}
Выберем точки на отрезке [a,b] с шагом $h = 0.2$.
\begin{center}
    $x_i = a + i   h$.
\end{center}
Для каждой точки $x_i$ найдём значение $f(x_i)$ и составим таблицу результатов (Таблица 1).
\begin{table}[h]
    \centering
    \begin{tabular}{|c|c|}
        \hline
        $x_i$ & $f(x_i)$\\
        \hline
        0,0 & 0,0000000000\\
        \hline
        0,2 & 0,2227025926\\
        \hline
        0,4 & 0,4283923805\\
        \hline
        0,6 & 0,6038561463\\
        \hline
        0,8 & 0,7421009541\\
        \hline
        1,0 & 0,8427006602\\
        \hline
        1,2 & 0,9103140831\\
        \hline
        1,4 & 0,9522852302\\
        \hline
        1,6 & 0,9763484001\\
        \hline
        1,8 & 0,9890906215\\
        \hline
        2,0 & 0,9953226447\\
        \hline
    \end{tabular}
    \caption*{\small{Таблица 1 - точки $x_i$ и значения разложения в ряд Тейлора $f(x_i)$}}
\end{table}

После нахождения значений разложения в ряд Тейлора в точках вычислим значение erf(x) при помощи 5 составных квадратурных формул. Для каждой формулы составим свою таблицу. В таблицах будут находится значения точки, для которой производились расчёты, значение разбиения в ряд Тейлора, значение найденного с помощью формулы интеграла в точке, модуль разницы между значениями найденного интеграла и разбиения, количества разбиений, которые пришлось совершить для нахождения значения интеграла с нужной точностью.
\begin{enumerate}[label = \arabic*.]
    \item {Правые прямоугольники:
        \begin{table}[h]
          \centering
          \begin{tabular}{|c|c|c|c|c|}
            \hline
            $x_i$ & $J_0(x_i)$ & $J_(x_i)$ & $\left|J_0(x_i) - J_N(x_i)\right|$ & $N$\\
            \hline
            0,0 & 0,0000000000 & 0,0000000000 & 0,0000000000 & 2\\
            \hline
            0,2 & 0,2227025926 & 0,2226983160 & 0,0000042766 & 1024\\
            \hline
            0,4 & 0,4283923805 & 0,4283596873 & 0,0000326931 & 1024\\
            \hline
            0,6 & 0,6038561463 & 0,6037563682 & 0,0000997782 & 1024\\
            \hline
            0,8 & 0,7421009541 & 0,7418920994 & 0,0002088547 & 1024\\
            \hline
            1,0 & 0,8427006602 & 0,8423525691 & 0,0003480911 & 1024\\
            \hline
            1,2 & 0,9103140831 & 0,9098084569 & 0,0005056262 & 1024\\
            \hline
            1,4 & 0,9522852302 & 0,9516219497 & 0,0006632805 & 1024\\
            \hline
            1,6 & 0,9763484001 & 0,9764841199 & 0,0001357198 & 1024\\
            \hline
            1,8 & 0,9890906215 & 0,9891686440 & 0,0000780225 & 1024\\
            \hline
            2,0 & 0,9953226447 & 0,9953628182 & 0,0000401735 & 1024\\
            \hline
          \end{tabular}
          \caption*{\small{Таблица 2 - таблица значений для формулы Правых прямоугольников}}
        \end{table}
        \newpage
    }
    \item {Центральные прямоугольники:
        \begin{table}[h]
          \centering
          \begin{tabular}{|c|c|c|c|c|}
            \hline
            $x_i$ & $J_0(x_i)$ & $J_(x_i)$ & $\left|J_0(x_i) - J_N(x_i)\right|$ & $N$\\
            \hline
            0,0 & 0,0000000000 & 0,0000000000 & 0,0000000000 & 2\\
            \hline
            0,2 & 0,2227025926 & 0,2227027565 & 0,0000001639 & 64\\
            \hline
            0,4 & 0,4283923805 & 0,4283923209 & 0,0000000596 & 256\\
            \hline
            0,6 & 0,6038561463 & 0,6038563848 & 0,0000002384 & 256\\
            \hline
            0,8 & 0,7421009541 & 0,7421010733 & 0,0000001192 & 512\\
            \hline
            1,0 & 0,8427006602 & 0,8427013755 & 0,0000007153 & 512\\
            \hline
            1,2 & 0,9103140831 & 0,9103139043 & 0,0000001788 & 512\\
            \hline
            1,4 & 0,9522852302 & 0,9522854686 & 0,0000002384 & 512\\
            \hline
            1,6 & 0,9763484001 & 0,9763489366 & 0,0000005364 & 256\\
            \hline
            1,8 & 0,9890906215 & 0,9890908003 & 0,0000001788 & 512\\
            \hline
            2,0 & 0,9953226447 & 0,9953227639 & 0,0000001192 & 256\\
            \hline
          \end{tabular}
          \caption*{\small{Таблица 3 - таблица значений для формулы Центральных прямоугольников}}
        \end{table}
    }
    \item {Формула Трапеций:
        \begin{table}[h]
          \centering
          \begin{tabular}{|c|c|c|c|c|}
            \hline
            $x_i$ & $J_0(x_i)$ & $J_(x_i)$ & $\left|J_0(x_i) - J_N(x_i)\right|$ & $N$\\
            \hline
            0,0 & 0,0000000000 & 0,0000000000 & 0,0000000000 & 2\\
            \hline
            0,2 & 0,2227025926 & 0,2229140997 & 0,0002115071 & 128\\
            \hline
            0,4 & 0,4283923805 & 0,4287676811 & 0,0003753006 & 512\\
            \hline
            0,6 & 0,6038561463 & 0,6043169498 & 0,0004608035 & 512\\
            \hline
            0,8 & 0,7421009541 & 0,7425656319 & 0,0004646778 & 512\\
            \hline
            1,0 & 0,8427006602 & 0,8431062102 & 0,0004055500 & 512\\
            \hline
            1,2 & 0,9103140831 & 0,9106265903 & 0,0003125072 & 512\\
            \hline
            1,4 & 0,9522852302 & 0,9525024891 & 0,0002172589 & 512\\
            \hline
            1,6 & 0,9763484001 & 0,9764841199 & 0,0001357198 & 512\\
            \hline
            1,8 & 0,9890906215 & 0,9891686440 & 0,0000780225 & 512\\
            \hline
            2,0 & 0,9953226447 & 0,9953628182 & 0,0000401735 & 512\\
            \hline
          \end{tabular}
          \caption*{\small{Таблица 4 - таблица значений для формулы Трапеций}}
        \end{table}
    }
    \item Формула Симпсона
        \begin{enumerate}
            \item {Вывод формулы Симпсона через интегральный полином Лагранжа:\\
            Формула для полинома Лагранжа:
            \begin{equation}
                L_n(x) = \sum_{i=0}^{n}f(x_i)\prod_{i \ne j, j = 0}^{n}\frac{x - x_j}{x_i - x_j}
            \end{equation}
            По трём узлам $(x_1 = a, x_2 = \dfrac{a+b}{2}, x_3 = b):
            L_2 = f(a)\left(\dfrac{x - \dfrac{a+b}{2}}{a - \dfrac{a+b}{2}}\right)\left(\dfrac{x - b}{a - b}\right)+\\
            f\left(\dfrac{a +b}{2}\right)\left(\dfrac{x-a}{\dfrac{a+b}{2} - a}\right)\left(\dfrac{x-b}{\dfrac{a+b}{2} - b}\right)+ f(b)\left(\dfrac{x - \dfrac{a+b}{2}}{b - \dfrac{a+b}{2}}\right)\left(\dfrac{x - b}{b - a}\right).$\\
            \hfill\break
            Проинтегрируем выражение по интервалу [a,b]:
            \begin{equation}
                \int\limits_{a}^{b}L_2(x)\mathrm{d}x = f(a)c_1 + f\left(\frac{a+b}{2}\right)c_2 + f(b)c_3
            \end{equation}
            где $c_1 = \dfrac{b-a}{6}, c_2 = \dfrac{2}{3}(b - a), c_3 = \dfrac{b-a}{6}.$\\
            \hfill\break
            Тогда:
            \begin{equation}
                \int\limits_{a}^{b}L_2(x)\mathrm{d}x = \frac{b-a}{6}\left(f(a) + 4f\left(\frac{a+b}{2}\right)+f(b)\right)
            \end{equation}
            }
            \item {Значения полученные для формулы Симпсона:
            \begin{table}[h]
            \centering
            \begin{tabular}{|c|c|c|c|c|}
                \hline
                $x_i$ & $J_0(x_i)$ & $J_(x_i)$ & $\left|J_0(x_i) - J_N(x_i)\right|$ & $N$\\
                \hline
                0,0 & 0,0000000000 & 0,0000000000 & 0,0000000000 & 2\\
                \hline
                0,2 & 0,2227025926 & 0,2227026075 & 0,0000000149 & 2\\
                \hline
                0,4 & 0,4283923805 & 0,4283923805 & 0,0000000000 & 4\\
                \hline
                0,6 & 0,6038561463 & 0,6038562059 & 0,0000000596 & 8\\
                \hline
                0,8 & 0,7421009541 & 0,7421009541 & 0,0000000000 & 8\\
                \hline
                1,0 & 0,8427006602 & 0,8427007794 & 0,0000001192 & 16\\
                \hline
                1,2 & 0,9103140831 & 0,9103139639 & 0,0000001192 & 16\\
                \hline
                1,4 & 0,9522852302 & 0,9522852302 & 0,0000000000 & 8\\
                \hline
                1,6 & 0,9763484001 & 0,9763483405 & 0,0000000596 & 16\\
                \hline
                1,8 & 0,9890906215 & 0,9890906215 & 0,0000000000 & 32\\
                \hline
                2,0 & 0,9953226447 & 0,9953221679 & 0,0000004768 & 32\\
                \hline
            \end{tabular}
            \caption*{\small{Таблица 5 - таблица значений для формулы Симпсона}}
            \end{table}
            }
        \end{enumerate}
    \item {Формула Гаусса:
        \begin{table}[h]
          \centering
          \begin{tabular}{|c|c|c|c|c|}
            \hline
            $x_i$ & $J_0(x_i)$ & $J_(x_i)$ & $\left|J_0(x_i) - J_N(x_i)\right|$ & $N$\\
            \hline
            0,0 & 0,0000000000 & 0,0000000000 & 0,0000000000 & 2\\
            \hline
            0,2 & 0,2227025926 & 0,2227025777 & 0,0000000149 & 2\\
            \hline
            0,4 & 0,4283923805 & 0,4283923209 & 0,0000000596 & 4\\
            \hline
            0,6 & 0,6038561463 & 0,6038560867 & 0,0000000596 & 8\\
            \hline
            0,8 & 0,7421009541 & 0,7421008945 & 0,0000000596 & 8\\
            \hline
            1,0 & 0,8427006602 & 0,8427007794 & 0,0000001192 & 16\\
            \hline
            1,2 & 0,9103140831 & 0,9103140235 & 0,0000000596 & 16\\
            \hline
            1,4 & 0,9522852302 & 0,9522851706 & 0,0000000596 & 8\\
            \hline
            1,6 & 0,9763484001 & 0,9763484001 & 0,0000000000 & 16\\
            \hline
            1,8 & 0,9890906215 & 0,9890905023 & 0,0000001192 & 32\\
            \hline
            2,0 & 0,9953226447 & 0,9953223467 & 0,0000002980 & 32\\
            \hline
          \end{tabular}
          \caption*{\small{Таблица 6 - таблица значений для формулы Гаусса}}
        \end{table}
    }
\end{enumerate}
\section{Выводы}
\hspace{5mm}Проделав все вычисления, можно сделать выводы, что более комплексные методы вычисления интеграла, как формула Гаусса и Симпсона, показыают наилучшие результаты за меньшее количество разбиений. В это же время худшие результаты вычисления показыают методы Правых прямоугольников и метод Трапеций, приводя к довольно большому значению ошибки.
\section{Листинг программы}
\lstinputlisting[language=C++]{./main.cpp}
\end{document}